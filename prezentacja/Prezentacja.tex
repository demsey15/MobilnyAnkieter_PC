\documentclass[a4paper,10pt]{beamer}
\usepackage[utf8]{inputenc}
\usepackage{polski}
\usepackage[OT4,T1]{fontenc}
\usepackage{amsmath}
\usepackage{amsthm}
\usepackage{graphicx}
\usepackage{ dsfont }
\usepackage{ amssymb }
\usepackage{enumerate}
\usepackage{tikz}

\usetheme{Warsaw}
\usecolortheme{beaver}

\newtheorem{defi}{Definicja}[subsection]
\newtheorem{uw}{Uwaga}[subsection]
\newtheorem{cel}{Cel}[subsection]
\newtheorem{tw}{Twierdzenie}[subsection]
\newtheorem{lem}{Lemat}[subsection]
\newtheorem{przyk}{Przykład}[subsection]
\newtheorem{alg}{Algorytm}[subsection]

\date{23 czerwca 2015}
\title{Bezpieczny ankieter}
\author[A. Bohonos, D. Demski, A. Mieldzioc]{Andrzej Bohonos, Dominik Demski, Adam Mieldzioc}

\begin{document}
		\begin{frame}
			\titlepage
		\end{frame}
		\begin{frame}{Agenda}
			\tableofcontents
		\end{frame}
		
		\section{Założenia projektu}
		\begin{frame}
		
		\end{frame}
	
		\section{Zakres wstępny i co zostało osiągnięte}
	
		\begin{frame}{Aplikacja desktopowa}
		\end{frame}
	
		\begin{frame}{Aplikacja mobilna - udało się zrealizować}
			\begin{enumerate}
				\item Stworzenie nowej ankiety. (UCM 1)
				\item	Przeprowadzenie ankiety. (UCM 2)
				\item	Wysłanie wyników przeprowadzonych ankiet do Głównego Systemu Ankiet. (UCM 3)
				\item	Ustawienie automatycznego wysyłania wyników ankiety do Głównego Systemu Ankiet. (UCM 4)
				\item	Pobranie ankiet z Głównego Systemu Ankiet. (UCM 5)
				\item	Ustalenie uprawnień Ankietera. (UCM 6)	
			\end{enumerate}
		\end{frame}
		
		\begin{frame}{Aplikacja mobilna - nie udało się zrealizować}
			\begin{enumerate}
				\item	Wykorzystanie istniejącej ankiety do stworzenia nowej.
				\item	Zmiana kolejności pytań w ankiecie.
				\item	Dodanie zdjęcia do ankiety jako części pytania.
				\item	Wyświetlenie statystyk dotyczących ankietera.
			    \item	Edytowanie pytań istniejącej już ankiety.
			\end{enumerate}
		\end{frame}
		
		\section{Architektura projektu}
		
		\begin{frame}{Model logiczny}
			
		\end{frame}
		
		\begin{frame}{Model wdrożeniowy}
			
		\end{frame}
		
		\section{Podział zadań}
		
		\begin{frame}{Andrzej}
			
		\end{frame}
		
		\begin{frame}{Adam}
			
		\end{frame}
		
		\begin{frame}{Dominik}
			
		\end{frame}
		
\end{document}