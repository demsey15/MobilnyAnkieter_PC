\documentclass[a4paper,10pt]{beamer}
\usepackage[utf8]{inputenc}
\usepackage{polski}
\usepackage[OT4,T1]{fontenc}
\usepackage{amsmath}
\usepackage{amsthm}
\usepackage{graphicx}
\usepackage{ dsfont }
\usepackage{ amssymb }
\usepackage{enumerate}
\usepackage{tikz}
\DeclareUnicodeCharacter{00A0}{ }
\usetheme{Warsaw}
\usecolortheme{beaver}

\newtheorem{defi}{Definicja}[subsection]
\newtheorem{uw}{Uwaga}[subsection]
\newtheorem{cel}{Cel}[subsection]
\newtheorem{tw}{Twierdzenie}[subsection]
\newtheorem{lem}{Lemat}[subsection]
\newtheorem{przyk}{Przykład}[subsection]
\newtheorem{alg}{Algorytm}[subsection]

\date{15 października 2015}
\title{Sprytny ankieter}
\author[A. Bohonos, D. Demski, A. Mieldzioc]{Andrzej Bohonos, Dominik Demski, Adam Mieldzioc}

\begin{document}
		\begin{frame}
			\titlepage
		\end{frame}
		\begin{frame}{Agenda}
			\tableofcontents
		\end{frame}
		
		\section{Cel i założenia projektu}
		\begin{frame}{Założenia projektu}
			Celem projektu jest stworzenie systemu umożliwiającego tworzenie i przeprowadzania ankiet. W skład systemu wchodzą:
			\begin{itemize}
				\item aplikacja mobilna dla systemu Android przeznaczona dla ankieterów,
				\item aplikacja desktopowa stworzona w języku Java z myślą o administrowaniu całością,
				\item serwer odgrywający rolę pośrednika.
			\end{itemize}
		\end{frame}
		
		\begin{frame}{Początkowy model wdrożeniowy}
				\pgfdeclareimage[width=11cm,height=7cm]{komponenty}{Diagramostateczny.png}
				\pgfuseimage{komponenty}
		\end{frame}
	
		\section{Co zostało osiągnięte}
	
		\begin{frame}{Aplikacja desktopowa - udało się zrealizować}
			\begin{enumerate}
				\item  Stworzenie nowej ankiety. 
				\item	Edycja pytań istniejącej ankiety. 
				\item	Zmiana statusu istniejącej ankiety. 
				\item	Wykorzystanie istniejącej ankiety do stworzenia nowej. 
				\item	Zmiana kolejności pytań w ankiecie.
				\item	Udostępnianie ankiety innym ankieterom. 
			\end{enumerate}
		\end{frame}
		
		\begin{frame}
			\begin{enumerate}
				\item  Rejestracja nowego ankietera. 
				\item	Zmiana statusu ankietera. 
				\item	Nadawanie ankieterom uprawnień do ankiety.
				\item	Przygotowanie list pracujących ankieterów.
				\item	Wyświetlanie rankingu ankieterów.
				\item	Wyświetlanie statystyk dotyczących ankietera.
				\item	Przygotowanie podstawowych statystyk dotyczących wypełniania ankiety.
			\end{enumerate}
		\end{frame}
			
	
		\begin{frame}{Aplikacja mobilna - udało się zrealizować}
			\begin{enumerate}
				\item Stworzenie nowej ankiety. 
				\item	Przeprowadzenie ankiety. 
				\item	Wysłanie wyników przeprowadzonych ankiet do Głównego Systemu Ankiet. 
				\item	Ustawienie automatycznego wysyłania wyników ankiety do Głównego Systemu Ankiet.
				\item	Pobranie ankiet z Głównego Systemu Ankiet. 
				\item	Ustalenie uprawnień Ankietera.	
			\end{enumerate}
		\end{frame}
		
		\section{Nowy zakres projektu}
			\begin{frame}{Nowy zakres}
				\begin{enumerate}
					\item Aplikacja desktopowa będzie umożliwiała tworzenie ankiet i przesyłanie ich bezpośrednio na urządzenie mobilne. Aplikacja będzie umożliwiała analizę zebranych wyników.
					\item Aplikacja mobilna będzie komunikowała się bezpośrednio z aplikacją desktopową.	
				\end{enumerate}
			\end{frame}
			\begin{frame}{Diagram wdrożeniowy}
				%Diagram wdrozeniowy - Andrzej zrobi i wyśle
			\end{frame}
			\begin{frame}{Do zrobienia - aplikacja desktopowa}
				\begin{enumerate}
					\item Zapisywanie wyników ankiet do pliku .csv. 
					\item Przygotowanie podstawowych statystyk.
					\item Baza danych przechowująca dane z zebranych ankiet.	
				\end{enumerate}
			\end{frame}
			\begin{frame}{Do zrobienia - aplikacja mobilna}
				\begin{enumerate}
					\item Wykorzystanie starej ankiety do stworzenia nowej.
					\item Poprawa tworenia ankiet.
					\item Komunikacja z komputerem.
					\item Przystosowanie aplikacji do używania jej na różnych urządzeniach.
				\end{enumerate}
			\end{frame}


\end{document}